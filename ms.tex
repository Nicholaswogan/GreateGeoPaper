\documentclass[12pt]{article}
\usepackage{tgtermes}
\usepackage[round]{natbib}
\usepackage{geometry}
\geometry{margin=1in}
\usepackage{setspace}
\doublespacing
\usepackage[hidelinks]{hyperref}

\begin{document}

\title{Lord Kelvin's debate over the age of the Earth: an early example of failed interdisciplinary science}
\author{Nicholas Wogan}
\maketitle

\begin{abstract}
  In the 1860s, Lord Kelvin used his understanding of physics and thermodynamics to estimate the age of the Earth. He brought these calculations to the attention of geologists, arguing against the uniformitarian view of an indefinitely old Earth and troubling Darwin, who wanted vast periods of history for his theory of natural selection. The ensuing debate, involving Lyell's geology, Darwin's biology, and Kelvin's physics, is an early attempt of interdisciplinary science. However, Kelvin's quantitative arguments for a young Earth were not entertained or understood by large portion of the geologic community for about a decade. Here, I argue that one reason this interdisciplinary science failed to proceed smoothly is because, at the time, there were big cultural divisions between the quantitative physical sciences and the more qualitative geologic and biological sciences. Today, some of the big unsolved problems in science will likely require collaborations between many different disciplines. To solve these big problems, perhaps we can learn from the failed 1860s debate over the age of the Earth, by striving to break down cultural and tribal barriers between different fields of study.
\end{abstract}

\section{Introduction}

\section{Kelvin, Lyell, and Darwin and the debate over the age of the Earth}

To piece together the following telling of the debate over the age of the Earth, I mostly draw from summaries in \citet{Lindley_2004}, \citet{Gould_1987} and \citet{Hallam_1989}, along with with a number of primary references from the 1800s.

Lord Kelvin's interest in the age of the Earth began in the early 1840s when he applied Fourier's theory of heat flow (i.e., heat flow by conduction) to various idealized systems. By 1844, he specifically applied his mathematics to Earth, assuming it to be a cooling conducting sphere with no internal heat flow. The mathematics did not yield physical solutions for an infinitely old Earth, which led Kelvin to conclude that Earth's age must be finite. Further, Kelvin argued, in his 1846 dissertation to Glasgow University, that he could calculate the age of the Earth if he had measurements of Earth's modern heat flow in the crust. However, at the time, heat flow measurements did not exist, so Kelvin's estimates for Earth's age based on its heat loss had to wait.

In the 1850s, lacking data for Earth's heat flow, Kelvin turned his attention to the related problem of the origin of the Sun's energy. He had two general hypotheses. His first idea was that the Sun's energy came from a stream of meteors colliding with it, converting kinetic energy to thermal energy. However, this hypothesis was ultimately dismantled by the French Astronomy Urbain Leverrier, who showed that Kelvin's meteor theory was incompatible with his observations Venus' and Mercury's orbits. If the Sun was gaining mass from meteors, then this would change Venus' and Mercury's orbits, but Leverrier found that their orbits were fairly stable over time. Kelvin's second idea for the origin of the Sun's heat, was that it may be explained by the slow conversion of gravitational energy into heat. In a 1862 magazine article, Kelvin argued that gravitational contraction could fuel the sun for between 20 and 500 million years.

In 1860, Kelvin finally got his hands on some heat flow measurements for Earth and applied his 1844 mathematics to estimate the time since Earth was completely molten. His calculations put Earth's age between 20 and 400 million years, a result he presented in 1862 to the Royal Society of Edinburgh. His presentation began with an attack on geologists: 

\begin{quote}
  For eighteen years it has pressed on my mind, that essential principles of Thermo-dynamics have been overlooked by those geologists who uncompromisingly oppose all paroxysmal hypotheses, and maintain not only that we have examples now before us, on the earth, of all the different actions by which its crust has been modified in geological history, but that these actions have never, or have not on the whole, been more violent in the past than they are at present.
\end{quote}
Here, Kelvin is attacking the geologic paradigm of uniformitarianism, which was most famously outlined in Charles Lyell's book \emph{Principle of Geology} \citep{Lyell_1833}, and adopted by many geologists during this era. Below, the next few paragraphs take a short detour to explain Lyell's uniformitarianism because it is necessary to understand Kelvin's 1960s debate with geologist's over the age of the Earth.

The subtitle of Lyell's \emph{Principles of Geology} gives a crude summary of the uniformitarianism philosophy: ``An attempt to explain the former changes of the Earth's surface, by reference to causes now in operation''. One of Lyell's arguments was uncontroversial, even by today's scientific standards: The laws of nature are uniform or constant through time. But Lyell took uniformity further, arguing that rate of geologic processes, like the erosion of a river valley or the building of a mountain, is roughly constant over time. Also, Lyell argued for an Earth that did not change qualitatively and has, more or less always looked and behaved the same. An implication of this philosophy is that Earth is extremely, or perhaps indefinitely, old.

Critically, Lyell makes clear that geology is a qualitative rather than a quantitative endeavor:

\begin{quote}
  The geologist myriads of ages were reckoned not by arithmetical computation, but by a train of physical events - a succession of phenomena in the animate and inanimate worlds - signs which convey to our minds more definite ideas than figures can do, of the immensity of time. \citep{Lyell_1833}
\end{quote}
To paraphrase, an immeasurably old Earth can be best understood by imagining the slow progression of Earth evolving over time, rather than ``arithmetical computation''.

Charles Darwin was among the many geologists to adopt some version of Lyell's uniformitarianism. Darwin recognizes that his theory of evolution by natural selection requires a vast history to explain all the change observed in the fossil record. To argue for an old Earth, Darwin cites Lyell's \emph{Principles of Geology} in the first edition of \emph{On the Origin of Species}, suggesting that the Earth is immeasurably old:

\begin{quote}
  He who can read Sir Charles Lyell's grand work on the Principles of Geology, which the future historian will recognise as having produced a revolution in natural science, yet does not admit how incomprehensibly vast have been the past periods of time, may at once close this volume. \citep{Darwin_1859}
\end{quote}
Furthermore, like Lyell, Darwin emphasizes that understanding the age of the Earth is more qualitative experience than a quantitative calculation:

\begin{quote}
  A man must for years examine for himself great piles of superimposed strata, and watch the sea at work grinding down old rocks and making fresh sediment, before he can hope to comprehend anything of the lapse of time, the monuments of which we see around us. \citep{Darwin_1859}
\end{quote}

With this context, we can now we can return to Kelvin's 1862 thermodynamic calculations for a 20 to 400 million year old Earth. Throughout the mid-1860s, Kelvin's arguments appear to have been largely ignored by the geologic community. Part of the reason why geologists ignored Kelvin was because within the paradigm of Lyell's uniformitarianism, there was a belief that the Earth was best understood through qualitative rather than quantitative observations (e.g., see the previous quote from \emph{Principles of Geology}). Furthermore geologists lacked the mathematical skills to evaluate Kelvin's claims. 

% For example, in regard to Kelvin's physical arguments about the age of the Earth, the geologist Andrew Ramsay said, ``I am as incapable of estimating and understanding the reasons which you physicists have for limiting geological time as you are incapable of understanding the geological reasons for our unlimited estimates.'' Kelvin responded that Ramsay could understand the physics, if he only put his ``mind to it''.

Darwin acknowledged Kelvin's arguments for a young Earth, and worried his claims were incompatible with natural selection if the rate of evolution was slow. In a letter to Alfred Russel Wallace in 1969, Darwin wrote, ``Thomson's views of the recent age of the world have been for some time one of my sorest troubles'' \citep{Marchant_1916}. Darwin further addresses the possibility of a young Earth in Chapter 9 of the 5th edition of \emph{On the Origin of Species}. 

\begin{quote}
  Take a narrow strip of paper, 83 feet 4 inches in length, and stretch it along the wall of a large hall; then mark off at one end the tenth of an inch. This tenth of an inch will represent one hundred years, and the entire strip a million years. \citep{Darwin_1869}
\end{quote}
This analogy seemed to make Darwin more comfortable with Kelvin's $\sim 100$ million year old Earth. Darwin also noted that biologists did not know the rate that species evolve, and so a young Earth with fast evolution was compatible with all the evidence.

\section{Lessons from history}

\section{Conclusions}

\bibliographystyle{plainnat}
\bibliography{bib}

\end{document}
