\documentclass[12pt]{article}
\usepackage{tgtermes}
\usepackage[round]{natbib}
\usepackage{geometry}
\geometry{margin=1in}
\usepackage{setspace}
\doublespacing
\usepackage[hidelinks]{hyperref}

\begin{document}

\title{The late 1800's debate over the age of the Earth: an early attempt of interdisciplinary science}
\author{Nicholas Wogan}
\maketitle

\section{Introduction}


Many of the big unsolved problems in science today will require an interdisciplinary approach to answer. For example, the unsolved problem of the origin of life on Earth can only be tackled using many different fields of research \citep{Hays_2015}. Geologic expertise is required to understand the earliest fossil records of life, organic chemists are needed to synthesize life-like replicating molecules, and atmospheric modelers are essential to estimate the plausible conditions on the early Earth under which life began. Another problem that requires broad expertise is projecting the impact of anthropogenic climate change on society. Any forecast will involve atmospheric scientists, glaciologists, economists, and ecologists among many other areas of knowledge. Given that important unanswered problems rely on an interdisciplinary approach, it is essential that scientists work together effectively even if they are embedded in different departments with ranging academic cultures.

One of the earliest attempts of interdisciplinary science is the debate over the age of the Earth in the late 1800s. The physicist Lord Kelvin used a thermodynamic model of heat flow in the crust to estimate Earth's age to be only $\sim 100$ million years old. He brought this calculation to the attention of geologists, who generally assumed the Earth was indefinitely old based on the paradigm of uniformitarianism. A relatively young Earth also troubled Darwin, who worried his theory of biological evolution needed more time than Kelvin allowed. The ensuing debate, involving physics, geology and biology was one of the earliest attempts of interdisciplinary science. However, the scientific discourse did not proceed smoothly. Geologists and physicists talked past each other, and neither discipline was capable of entertaining evidence outside their domain of expertise.

The purpose of this paper is to attempt to shed light on why interdisciplinary science failed during the late 1800s debate over the age of the Earth. Furthermore, I use this historical failure to argue how interdisciplinary science might be best carried out in modern research. To achieve these goals, I first summarize the late 1800s discourse over Earth's age (Section \ref{sec:2}). Next, in Section \ref{sec:3}, I provide possible reasons for why this early attempt of interdisciplinary science was unsuccessful, and what lessons might be drawn from history to improve modern interdisciplinary work. Finally, in Section \ref{sec:4}, I summarize and conclude.

\section{Kelvin, Lyell, and Darwin and the debate over the age of the Earth} \label{sec:2}

The following summary of the debate over the age of the Earth draws from \citet{Lindley_2004}, \citet{Gould_1987} and \citet{Hallam_1989}, along with with a number of primary references from the 1800s.

Lord Kelvin's interest in the age of the Earth began in the early 1840s when he applied Fourier's theory of heat flow (i.e., heat flow by conduction) to various idealized systems. By 1844, he specifically applied his mathematics to Earth, assuming it to be a cooling conducting half-space with no internal heat flow. The mathematics did not yield physical solutions for an infinitely old Earth, which led Kelvin to conclude that Earth's age must be finite. Further, Kelvin argued, in his 1846 dissertation to Glasgow University, that he could calculate the age of the Earth with measurements of Earth's modern heat flow in the crust. However, at the time, heat flow measurements did not exist, so Kelvin's estimates for Earth's age based on these calculations had to wait.

In the 1850s, lacking data for Earth's heat flow, Kelvin turned his attention to the related problem of the origin of the Sun's energy. He had two general hypotheses. His first idea was that the Sun's energy came from a stream of meteors colliding with it, converting kinetic energy to thermal energy. However, this hypothesis was ultimately dismantled by the French astronomer Urbain Leverrier, who showed that Kelvin's meteor theory was incompatible with his observations Venus and Mercury's orbits. If the Sun was gaining mass from meteors, then this would change Venus and Mercury's orbits, but Leverrier found that their orbits were fairly stable over time. Kelvin's second idea for the origin of the Sun's heat, was that it may be explained by the slow conversion of gravitational energy into heat. In a 1862 magazine article, Kelvin argued that gravitational contraction could fuel the sun for between 20 and 500 million years.

In 1860, Kelvin finally got his hands on some heat flow measurements for Earth's crust and applied his 1844 mathematics to estimate the time since Earth was completely molten. His calculations put Earth's age between 20 and 400 million years, a result he presented in 1862 to the Royal Society of Edinburgh. His presentation began with an attack on geologists: 

\begin{quote}
  For eighteen years it has pressed on my mind, that essential principles of Thermo-dynamics have been overlooked by those geologists who uncompromisingly oppose all paroxysmal hypotheses, and maintain not only that we have examples now before us, on the earth, of all the different actions by which its crust has been modified in geological history, but that these actions have never, or have not on the whole, been more violent in the past than they are at present.
\end{quote}
Here, Kelvin is criticizing the geologic paradigm of uniformitarianism, which was most famously outlined in Charles Lyell's book \emph{Principle of Geology} \citep{Lyell_1833}, and adopted by many geologists during this era. Below, the next few paragraphs take a short detour to explain Lyell's uniformitarianism because it is necessary to understand Kelvin's 1960s debate with geologist's over the age of the Earth.

The subtitle of Lyell's \emph{Principles of Geology} gives a crude summary of the uniformitarianism philosophy: ``An attempt to explain the former changes of the Earth's surface, by reference to causes now in operation''. One of Lyell's arguments was uncontroversial, even by today's scientific standards: The laws of nature are uniform or constant through time. But Lyell took uniformity further, arguing that rate of geologic processes, like the erosion of a river valley or the building of a mountain, are roughly constant over time. Also, Lyell argued for an Earth that has not changed qualitatively and has more or less always looked and behaved the same. An implication of this philosophy is that Earth is extremely old.

Critically, Lyell makes clear that geology is a qualitative rather than a quantitative endeavor:

\begin{quote}
  The geologist myriads of ages were reckoned not by arithmetical computation, but by a train of physical events - a succession of phenomena in the animate and inanimate worlds - signs which convey to our minds more definite ideas than figures can do, of the immensity of time. \citep{Lyell_1833}
\end{quote}
To paraphrase, an immeasurably old Earth can be best understood by imagining the slow progression of Earth evolving over time, rather than ``arithmetical computation''.

Charles Darwin was among the many geologists to adopt some version of Lyell's uniformitarianism. Darwin recognizes that his theory of evolution by natural selection requires a vast history to explain all the change observed in the fossil record. To argue for an old Earth, Darwin cites Lyell's \emph{Principles of Geology} in the first edition of \emph{On the Origin of Species}, suggesting that the Earth is very old:

\begin{quote}
  He who can read Sir Charles Lyell's grand work on the Principles of Geology, which the future historian will recognise as having produced a revolution in natural science, yet does not admit how incomprehensibly vast have been the past periods of time, may at once close this volume. \citep{Darwin_1859}
\end{quote}
Furthermore, like Lyell, Darwin emphasizes that understanding the age of the Earth is more qualitative experience than a quantitative calculation:

\begin{quote}
  A man must for years examine for himself great piles of superimposed strata, and watch the sea at work grinding down old rocks and making fresh sediment, before he can hope to comprehend anything of the lapse of time, the monuments of which we see around us. \citep{Darwin_1859}
\end{quote}

With this context, we can now we can return to Kelvin's 1862 thermodynamic calculations for a 20 to 400 million year old Earth. Throughout the mid-1860s, Kelvin's arguments appear to have been largely ignored by the geologic community. Part of the reason why geologists ignored Kelvin was because within the paradigm of Lyell's uniformitarianism, there was a belief that the Earth was best understood through qualitative rather than quantitative observations (e.g., see the previous quote from \emph{Principles of Geology}). Furthermore geologists lacked the mathematical skills to evaluate Kelvin's claims. This dynamic is illustrated by an interaction between Kelvin and the prominent geologist Andrew Ramsay at a 1867 conference in Dundee, Scotland. In response to Kelvin's calculations, Ramsay said, ``I am as incapable of estimating and understanding the reasons which you physicists have for limiting geological time as you are incapable of understanding the geological reasons for our unlimited estimates.''

Darwin was an exception, because he acknowledged Kelvin's arguments for a young Earth, and worried Kelvin's claims were incompatible with natural selection if the rate of evolution was slow. In a letter to Alfred Russel Wallace in 1969, Darwin wrote, ``Thomson's views of the recent age of the world have been for some time one of my sorest troubles'' \citep{Marchant_1916}. Darwin repeats his concern of Kelvin's work in the fifth and sixth editions of \emph{On the Origin of Species} \citep{Darwin_1869,Darwin_1872}. However, he also notes that biologists have only observed changes in species in the stratigraphic record, and have not quantified the rate of evolution measured by years. Therefore, according to Darwin, a young Earth with quick evolution is compatible with all the evidence. Furthermore, Darwin is generally skeptical of a physics-based calculation and wonders if Kelvin's calculation has bigger uncertainty than portrayed: ``...many philosophers are not as yet willing to admit that we know enough of the constitution of the universe and of the interior of our globe to speculate with safety on its past duration'' \citep{Darwin_1872}.

Through the late 1960s Kelvin published several more papers, repeating his 1962 arguments. Ultimately, the biologist Thomas Huxley used his presidential address to the London Geologic society to directly challenge Kelvin. He brought up Kelvin's changing theories for the Sun's heat and suggested that Kelvin's thermodynamic calculations of the cooling Earth had more uncertainty than Kelvin portrayed. Huxley also noted that no one believed strictly in Lyell's uniformitarianism and that geologists were generally comfortable with a 100-million-year-old Earth. However, this statement was not completely true because Darwin was clearly concerned based on his letters to Wallace. To conclude, Huxley stated: 

\begin{quote}
  And the critical examination of the grounds upon which the very grave charge of opposition to the principles of Natural Philosophy has been brought against us rather shows that we have exercised a wise discrimination in declining to meddle with our foundations at the bidding of the first passer-by who fancies our house is not so well built as it might be.
\end{quote}
Here, Kelvin is labeled a ``passer-by'' that is not part of the geologic community. Two months later Kelvin responded to Huxley at the Geological Society of Glasgow. A portion of Kelvin's remarks specifically addressed Huxley's final sentence:

\begin{quote}
  I cannot pass from Professor Huxley's last sentence without asking, who are the occupants of ``our house,'' and who is the ``passer-by''? Is geology not a branch of physical science? Are investigations experimental and mathematical, of underground temperature, not to be regarded as an integral part of geology? ... For myself, I am anxious to be regarded by geologists, not as a mere passer-by, but as one constantly interested in their grand subject, and anxious, in any way, however slight, to assist them in their search for truth.
\end{quote}

At this point in the debate, the physicist Peter Tait decided to defend his friend Kelvin by attacking Huxley along with the rest of the geologic community. Tait's remarks were full of arrogance, suggesting that the field of physics was superior to geology, and that they should be grateful for the help of a valuable mathematician. Furthermore, without any detailed justification, Tait claimed to prefer a 10 to 15-million-year-old Earth rather than Kelvin's median guess of 100 million years. Huxley did not formally respond to Tait's disparagement.

Kelvin's calculations for a 100-million-year-old Earth were of course wrong. The biggest problem was that he assumed that conduction was the only heat transfer mechanism in the Earth, but heat transfer is dominated by convection through most of the mantle. John Perry, a former student of Kelvin's, pointed out this flaw in Kelvin's original calculations in 1895 \citep{England_2007}. Perry's alternative model, which included convection, suggested a 10 billion-year-old Earth, a value much closer to the true value of 4.56 billion. Kelvin was somewhat receptive to Perry's arguments stating ``it is quite possible I should have put the superior limit a good deal higher, perhaps 4,000 [million years] instead of 400 [million years].'' However, despite this acknowledgement, Kelvin delivered a lecture in 1897 which repeated his 1862 arguments for a 100-million-year-old Earth without mention of Perry's work. This was Kelvin's last public comment about the age of the Earth before his death in 1907.

\section{A reason why late 1800s interdisciplinary science was unsuccessful and implications for modern research} \label{sec:3}

The debate over the age of the Earth was one of the earliest attempts of interdisciplinary science. The problem involved Kelvin's thermodynamic arguments, the geologic record, as well as Darwin's theory of biological evolution. However, this interdisciplinary experiment was not very successful. For years after his initial 1862 arguments for a young Earth, Kelvin was unable to convince geologists to take his calculations seriously. Also, Kelvin seemed to value his thermodynamics far more than evidence preserved in the stratigraphic record. Why couldn't any physicist, geologist or biologist consider all the evidence (geologic, or physics-based) to estimate Earth's age?

There are likely many reasons why interdisciplinary science failed. I believe one important reason was that, at the time, different disciplines were culturally tied to specific kinds of scientific evidence. Thanks to Lyell's paradigm of uniformitarianism, the late 1800s geologist took a qualitative approach, studying the Earth by solely examining piles of strata. This method for understanding the world was part of what it meant to be a geologist. In contrast, a physicist was defined by a mathematical or quantitative approach. Given this cultural division between the disciplines, no researcher was in a position to consider all evidence on equal footing.




\section{Conclusions} \label{sec:4}

In the 1860s, Lord Kelvin used his understanding of physics and thermodynamics to estimate the age of the Earth. He brought these calculations to the attention of geologists, arguing against the uniformitarian view of an indefinitely old Earth and troubling Darwin, who wanted vast periods of history for his theory of natural selection. The ensuing debate, involving Lyell's geology, Darwin's biology, and Kelvin's physics, is an early attempt of interdisciplinary science. However, Kelvin's quantitative arguments for a young Earth were not entertained or understood by large portion of the geologic community for about a decade. Here, I argue that one reason this interdisciplinary science failed to proceed smoothly is because, at the time, there were big cultural divisions between the quantitative physical sciences and the more qualitative geologic and biological sciences. Today, some of the big unsolved problems in science will likely require collaborations between many different disciplines. To solve these big problems, perhaps we can learn from the failed 1860s debate over the age of the Earth, by striving to break down cultural and tribal barriers between different fields of study.

\bibliographystyle{plainnat}
\bibliography{bib}

\end{document}
